\section{Messungen auf AMD-GPUs}
\label{amd}

\subsection{Verwendete Hard- und Software}

Die hier gezeigten Benchmark-Ergebnisse wurden auf einem Rechner mit der
folgenden Hardware gemessen:

\begin{itemize}
    \item CPU: AMD Ryzen Threadripper 1950X
          \begin{itemize}
              \item \num{16} Kerne
              \item \num{32} virtuelle Kerne
              \item maximaler Takt: \SI{3,4}{\giga\hertz}
          \end{itemize}
    \item GPU: AMD Radeon RX Vega 64
          \begin{itemize}
              \item \num{64} Multiprozessoren
              \item \num{64} Kerne pro Multiprozessor (insgesamt \num{4096}
                    Kerne)
              \item maximaler Takt: \SI{1536}{\mega\hertz}
              \item \SI{8}{\gibi\byte} HBM2-Speicher
              \item Speicherbusbreite: \SI{2048}{\bit}
              \item Speicherbandbreite: \SI{483,3}{\gibi\byte\per\second}
              \item Speichertakt: \SI{945}{\mega\hertz}
              \item kein ECC
          \end{itemize}
      \item RAM: \SI{64}{\gibi\byte}
\end{itemize}

Das verwendete Betriebssystem war Ubuntu 16.04 mit der Linux-Kernel-Version
4.15. Für die \gls{gpgpu}-Programmierung kamen die mit der \gls{rocm}-Version
2.1.96 mitgelieferten HIP- und HC-Implementierungen zum Einsatz.

\subsection{zcopy}

\subsubsection{Vorüberlegungen}

Die in Abschnitt~\ref{nvidia:zcopy:vorueberlegungen} für NVIDIA-GPUs gemachten
Überlegungen können nicht ohne Anpassungen für AMD-GPUs übernommen werden, da
sich die Multiprozessor-Architekturen unterscheiden.

Jeder Multiprozessor einer Vega-GPU besteht aus vier \gls{simd}-Einheiten. Eine
\gls{simd}-Einheit verfügt über 16 Vektor-ALUs (VALUs), die synchron auf jeweils
einem \textit{work-item} arbeiten. Die vier \gls{simd}-Einheiten arbeiten
parallel eine \textit{wavefront} ab, die 64 \textit{work-items} umfasst.

Anders ausgedrückt kann jeder Multiprozessor genau 64 Threads abarbeiten, wobei
diese auf dem Multiprozessor noch einmal in Gruppen von 16 Threads aufgeteilt
werden. Ausgehend von NVIDIAs \textit{half-warps} könnte man diese Gruppen auch
als \textit{quarter-wavefronts} bezeichnen, wenngleich dieser Begriff offiziell
nicht existiert.

Die Multiprozessoren verfügen über einen L1-Cache, der eine Cacheline-Größe von
64 Byte aufweist. Um jede \gls{simd}-Einheit effizient zu befüllen, muss jeder
Thread genau 4 Bytes lesen, was einem \texttt{float} entspräche. Um den
Speicher-Controller besser auszulasten, wird dieser Wert auf 8 Byte pro Thread
erhöht (Datentyp \texttt{hc::short\_vector::float\_2}). Die transportierte
Datenmenge pro Wavefront beträgt somit $8 \cdot 64 = 512$ Bytes und entspricht
damit der der CUDA-Variante ($16 \cdot 32 = 512$), bei der ein
\texttt{float4}-Vektor verwendet wurde.

\subsubsection{Messmethoden}
\label{amd:zcopy:methoden}

Die Messmethoden entsprechen dem in Abschnitt~\ref{nvidia:zcopy:methoden}
für NVIDIA-GPUs geschilderten Vorgehen.

Die im Quelltext~\ref{amd:zcopy:befehle} zeigen die verwendeten Compiler-Flags
sowie die Festsetzung der Multiprozessor-Taktrate.

\begin{code}
    \begin{minted}[fontsize=\small]{bash}
# HC-Compiler
hcc `hcc-config --cxxflags --ldflags` -O3 -std=c++17 \
  -amdgpu-target=gfx900

# HIP-Compiler (-amdgpu-target wird nicht unterstützt)
hipcc -O3 -std=c++17

# Taktrate
rocm-smi --setsclk 7
    \end{minted}
    \caption{Compiler-Flags und Taktrate für zcopy}
    \label{amd:zcopy:befehle}
\end{code}

\subsubsection{Ergebnisse}

Wie das Ergebnis für den kombinierten Lese- und Schreibvorgang zeigt (siehe
Abbildung~\ref{amd:zcopy:rw}), stagniert die Bandbreite der \gls{hc}-Kernel mit
steigender Tile-Zahl. Auch scheint eine mittlere Anzahl an Threads pro Tile
vorteilhaft sein.

Ein interessanter Effekt ist für den reinen Schreibvorgang zu beobachten (siehe
Abbildung~\ref{amd:zcopy:w}): Während Tiles mit \num{64} oder \num{256} Threads
eine nahezu konstante Bandbreite ermöglichen, brechen alle anderen Größen stark
ein, bevor sie sich mit steigender Tile-Zahl wieder in die Richtung des Maximums
bewegen. Eine Erklärung lässt sich dafür nur schwer finden, da es keine auf den
ersten Blick schlüssige Begründung für das Verhalten der \num{128}er-Tiles gibt,
die sich mit dem Verhalten der großen Tiles in Einklang bringen lässt. Denkbar
ist ein Fehler im Scheduler oder eine in dieser Konfiguration schlechte
Auslastung des Speichercontrollers; hier wären weitergehende Untersuchungen
erforderlich.

Die \gls{hip}-Variante verhält sich wie die \gls{hc}-Implementierung, die
Ergebnisse sind dieser Arbeit in Abschnitt~\ref{anhang:hip:amdzcopyfig}
angehängt.

\begin{figure}
    \centering
    \begin{tikzpicture}
        \begin{axis}[
            title = {zcopy -- Lesen + Schreiben -- Vega 64},
            xlabel = {Tiles pro Multiprozessor},
            ylabel = {Bandbreite [\si{\gibi\byte\per\second}]},
            xmode = log,
            log basis x = 2,
            xmin = 1, xmax = 4096,
            xticklabel = {\xinttheiexpr2^\tick\relax},
            log ticks with fixed point,
            ymajorgrids = true,
            xmajorgrids = true,
            grid style = dashed,
            legend cell align = left,
            legend pos = south east,
            no markers,
            every axis plot/.append style = {very thick},
            width = 0.75\textwidth,
            scale only axis,
            cycle list name = exotic,
            /pgf/number format/.cd, use comma
        ]
            \addplot table [x = blocks_per_sm, y = throughput-64,
                            col sep = semicolon]
                           {data/zcopy-amd-hc-vega64-rw.csv};
            \addlegendentry{$\text{Tile-Größe} = 64$} 

            \addplot table [x = blocks_per_sm, y = throughput-128,
                            col sep = semicolon]
                           {data/zcopy-amd-hc-vega64-rw.csv};
            \addlegendentry{$\text{Tile-Größe} = 128$} 

            \addplot table [x = blocks_per_sm, y = throughput-256,
                            col sep = semicolon]
                           {data/zcopy-amd-hc-vega64-rw.csv};
            \addlegendentry{$\text{Tile-Größe} = 256$} 

            \addplot table [x = blocks_per_sm, y = throughput-512,
                            col sep = semicolon]
                           {data/zcopy-amd-hc-vega64-rw.csv};
            \addlegendentry{$\text{Tile-Größe} = 512$} 

            \addplot table [x = blocks_per_sm, y = throughput-1024,
                            col sep = semicolon]
                           {data/zcopy-amd-hc-vega64-rw.csv};
            \addlegendentry{$\text{Tile-Größe} = 1024$} 
        \end{axis}
    \end{tikzpicture}
    \caption{Bandbreite der Vega 64 (Lesen und Schreiben, HC)}
    \label{amd:zcopy:rw}
\end{figure}

\begin{figure}
    \centering
    \begin{tikzpicture}
        \begin{axis}[
            title = {zcopy -- Schreiben -- Vega 64},
            xlabel = {Tiles pro Multiprozessor},
            ylabel = {Bandbreite [\si{\gibi\byte\per\second}]},
            xmode = log,
            log basis x = 2,
            xmin = 1, xmax = 4096,
            xticklabel = {\xinttheiexpr2^\tick\relax},
            log ticks with fixed point,
            ymajorgrids = true,
            xmajorgrids = true,
            grid style = dashed,
            legend cell align = left,
            legend pos = south east,
            no markers,
            every axis plot/.append style = {very thick},
            width = 0.75\textwidth,
            scale only axis,
            cycle list name = exotic,
            /pgf/number format/.cd, use comma
        ]
            \addplot table [x = blocks_per_sm, y = throughput-64,
                            col sep = semicolon]
                           {data/zcopy-amd-hc-vega64-w.csv};
            \addlegendentry{$\text{Tile-Größe} = 64$} 

            \addplot table [x = blocks_per_sm, y = throughput-128,
                            col sep = semicolon]
                           {data/zcopy-amd-hc-vega64-w.csv};
            \addlegendentry{$\text{Tile-Größe} = 128$} 

            \addplot table [x = blocks_per_sm, y = throughput-256,
                            col sep = semicolon]
                           {data/zcopy-amd-hc-vega64-w.csv};
            \addlegendentry{$\text{Tile-Größe} = 256$} 

            \addplot table [x = blocks_per_sm, y = throughput-512,
                            col sep = semicolon]
                           {data/zcopy-amd-hc-vega64-w.csv};
            \addlegendentry{$\text{Tile-Größe} = 512$} 

            \addplot table [x = blocks_per_sm, y = throughput-1024,
                            col sep = semicolon]
                           {data/zcopy-amd-hc-vega64-w.csv};
            \addlegendentry{$\text{Tile-Größe} = 1024$} 
        \end{axis}
    \end{tikzpicture}
    \caption{Bandbreite der Vega 64 (Schreiben, HC)}
    \label{amd:zcopy:w}
\end{figure}

Ein Leistungsunterschied besteht in der erreichten Bandbreite: \gls{hip} ist
hier messbar schneller, wenn auch nur um wenige \si{\giga\byte\per\second}. Dies
ist sowohl für den kombinierten Lese- und Schreib-Kernel als auch den reinen
Schreibvorgang der Fall, wie die Abbildungen~\ref{amd:zcopy:vergleichrw} und
\ref{amd:zcopy:vergleichw} zeigen.

\begin{figure}
    \centering
    \begin{tikzpicture}
        \begin{axis}[
            title = {zcopy -- Lesen+Schreiben -- Vega 64},
            xlabel = {Tiles pro Multiprozessor},
            ylabel = {Bandbreite [\si{\gibi\byte\per\second}]},
            xmode = log,
            log basis x = 2,
            xmin = 1, xmax = 4096,
            xticklabel = {\xinttheiexpr2^\tick\relax},
            log ticks with fixed point,
            ymajorgrids = true,
            xmajorgrids = true,
            grid style = dashed,
            legend cell align = left,
            legend pos = south east,
            no markers,
            every axis plot/.append style = {very thick},
            width = 0.75\textwidth,
            scale only axis,
            cycle list name = exotic,
            /pgf/number format/.cd, use comma
        ]
            \addplot table [x = blocks_per_sm, y = throughput-256,
                            col sep = semicolon]
                           {data/zcopy-amd-hc-vega64-rw.csv};
            \addlegendentry{HC} 

            \addplot table [x = blocks_per_sm, y = throughput-256,
                            col sep = semicolon]
                           {data/zcopy-amd-hip-vega64-rw.csv};
            \addlegendentry{HIP} 
        \end{axis}
    \end{tikzpicture}
    \caption{Bandbreite der Vega 64 (256er-Blöcke, Lesen+Schreiben)}
    \label{amd:zcopy:vergleichrw}
\end{figure}

\begin{figure}
    \centering
    \begin{tikzpicture}
        \begin{axis}[
            title = {zcopy -- Schreiben -- Vega 64},
            xlabel = {Tiles pro Multiprozessor},
            ylabel = {Bandbreite [\si{\gibi\byte\per\second}]},
            xmode = log,
            log basis x = 2,
            xmin = 1, xmax = 4096,
            xticklabel = {\xinttheiexpr2^\tick\relax},
            log ticks with fixed point,
            ymajorgrids = true,
            xmajorgrids = true,
            grid style = dashed,
            legend cell align = left,
            legend pos = south east,
            no markers,
            every axis plot/.append style = {very thick},
            width = 0.75\textwidth,
            scale only axis,
            cycle list name = exotic,
            /pgf/number format/.cd, use comma
        ]
            \addplot table [x = blocks_per_sm, y = throughput-256,
                            col sep = semicolon]
                           {data/zcopy-amd-hc-vega64-w.csv};
            \addlegendentry{HC} 

            \addplot table [x = blocks_per_sm, y = throughput-256,
                            col sep = semicolon]
                           {data/zcopy-amd-hip-vega64-w.csv};
            \addlegendentry{HIP} 
        \end{axis}
    \end{tikzpicture}
    \caption{Bandbreite der Vega 64 (256er-Blöcke, Schreiben)}
    \label{amd:zcopy:vergleichw}
\end{figure}

Wie die Abbildung~\ref{amd:zcopy:peak} zeigt, lassen sich etwa 90\% der
theoretisch möglichen Bandbreite auch praktisch nutzen.

\begin{figure}
    \centering
    \begin{tikzpicture}
        \begin{axis}[
            title = {zcopy -- Vega 64},
            xlabel = {Tiles pro Multiprozessor},
            ylabel = {Bandbreite [\si{\gibi\byte\per\second}]},
            xmode = log,
            log basis x = 2,
            xmin = 1, xmax = 4096,
            ymin = 180, ymax = 500,
            xticklabel = {\xinttheiexpr2^\tick\relax},
            log ticks with fixed point,
            ymajorgrids = true,
            xmajorgrids = true,
            grid style = dashed,
            legend cell align = left,
            legend pos = south east,
            no markers,
            every axis plot/.append style = {very thick},
            width = 0.75\textwidth,
            scale only axis,
            cycle list name = exotic,
            extra y ticks = 483.8,
            extra y tick labels = {$483.8$},
            /pgf/number format/.cd, use comma
        ]
            \addplot table [x = blocks_per_sm, y = throughput-256,
                            col sep = semicolon]
                           {data/zcopy-amd-hc-vega64-rw.csv};
            \addlegendentry{Lesen+Schreiben} 

            \addplot table [x = blocks_per_sm, y = throughput-256,
                            col sep = semicolon]
                           {data/zcopy-amd-hc-vega64-w.csv};
            \addlegendentry{Schreiben} 

            \addplot table [x = blocks_per_sm, y = peak, col sep = semicolon]
                           {data/zcopy-amd-vega64-peak.csv};
            \addlegendentry{Peak}
        \end{axis}
    \end{tikzpicture}
    \caption{Theoretische und praktische Bandbreite der Vega 64 (HC,
             256er-Tiles)}
    \label{amd:zcopy:peak}
\end{figure}

\subsection{Reduction}

\subsubsection{Implementierung}

Die Implementierungen des Reduktionskernels sind dieser Arbeit angehängt und
befinden sich in den Quelltexten~\ref{anhang:hc:reduction} (\gls{hc}) und
\ref{anhang:hip:reduction} (\gls{hip}).

\subsubsection{Messmethoden}

Die Messmethoden entsprechen den in Abschnitt~\ref{nvidia:reduction:methoden}
für die NVIDIA-Implementierungen geschilderten, die Compiler-Flags und
GPU-Einstellungen sind dieselben wie für den zcopy-Benchmark (siehe
Abschnitt~\ref{amd:zcopy:methoden}).

\subsubsection{Ergebnisse}

Wie in der Abbildung~\ref{amd:reduction:hc} zu sehen ist, sind 256 Threads die
beste Tile-Größe der HC-Implementierung, während zwei Tiles dieser Größe die
beste Anzahl pro Multiprozessor darstellen. Dies gilt auch für HIP (siehe
Abbildung~\ref{anhang:amd:reduction:hip}).

Im direkten Vergleich der Spracherweiterungen wird sichtbar, dass sich die
zcopy-Ergebnisse bei der Reduktion bestätigen lassen. Wieder erreichen \gls{hc}
und \gls{hip} eine ähnliche Bandbreite, wobei \gls{hip} konstant und messbar
bessere Werte erreicht. Beide Implementierungen kommen außerdem sehr nah an die
tatsächlich erreichbare Bandbreite heran, die im zcopy-Benchmark ermittelt wurde
(siehe Abbildung~\ref{amd:reduction:peak}).

\begin{figure}
    \centering
    \begin{tikzpicture}
        \begin{axis}[
            title = {Reduction -- Vega 64 -- HC},
            xlabel = {Tiles pro Multiprozessor},
            ylabel = {Bandbreite [\si{\gibi\byte\per\second}]},
            xmode = log,
            log basis x = 2,
            xmin = 2, xmax = 1024,
            xticklabel = {\xinttheiexpr2^\tick\relax},
            log ticks with fixed point,
            ymajorgrids = true,
            xmajorgrids = true,
            grid style = dashed,
            legend cell align = left,
            legend pos = south east,
            no markers,
            every axis plot/.append style = {very thick},
            width = 0.75\textwidth,
            scale only axis,
            cycle list name = exotic,
            /pgf/number format/.cd, use comma
        ]
            \addplot table [x = tiles, y = dim64,
                            col sep = semicolon]
                           {data/reduce-amd-hc-vega64.csv};
            \addlegendentry{Tile-Größe = 64} 

            \addplot table [x = tiles, y = dim128,
                            col sep = semicolon]
                           {data/reduce-amd-hc-vega64.csv};
            \addlegendentry{Tile-Größe = 128} 

            \addplot table [x = tiles, y = dim256,
                            col sep = semicolon]
                           {data/reduce-amd-hc-vega64.csv};
            \addlegendentry{Tile-Größe = 256} 

            \addplot table [x = tiles, y = dim512,
                            col sep = semicolon]
                           {data/reduce-amd-hc-vega64.csv};
            \addlegendentry{Tile-Größe = 512} 

            \addplot table [x = tiles, y = dim1024,
                            col sep = semicolon]
                           {data/reduce-amd-hc-vega64.csv};
            \addlegendentry{Tile-Größe = 1024} 
        \end{axis}
    \end{tikzpicture}
    \caption{Reduction-Bandbreite der Vega 64 (HC)}
    \label{amd:reduction:hc}
\end{figure}

\begin{figure}
    \centering
    \begin{tikzpicture}
        \begin{axis}[
            title = {Reduction -- Vega 64},
            xlabel = {$n$},
            ylabel = {Bandbreite [\si{\gibi\byte\per\second}]},
            xmode = log,
            log basis x = 2,
            xmin = 65536, xmax = 134217728,
            ymajorgrids = true,
            xmajorgrids = true,
            grid style = dashed,
            legend cell align = left,
            legend pos = south east,
            no markers,
            every axis plot/.append style = {very thick},
            width = 0.75\textwidth,
            scale only axis,
            cycle list name = exotic,
            /pgf/number format/.cd, use comma
        ]
            \addplot table [x = n, y = HC,
                            col sep = semicolon]
                           {data/reduce-amd-vega64.csv};
            \addlegendentry{HC} 

            \addplot table [x = n, y = HIP,
                            col sep = semicolon]
                           {data/reduce-amd-vega64.csv};
            \addlegendentry{HIP} 

            \addplot table [x = , y = Peak, col sep = semicolon]
                           {data/reduce-amd-vega64.csv};
            \addlegendentry{Peak}
        \end{axis}
    \end{tikzpicture}
    \caption{Reduction-Bandbreite der Vega 64 (zwei 256er-Tiles pro
             Multiprozessor)}
    \label{amd:reduction:peak}
\end{figure}

\subsection{N-Body}

\subsubsection{Implementierung}

Die theoretische Funktion~\ref{methoden:nbody:gpu:bodybodyinteraction}, die die
Interaktion zwischen zwei Körpern berechnet, wurde direkt in beiden Sprachen
umgesetzt. Durch den Einsatz von \gls{fma}-Operationen werden die benötigten
\gls{flops} für die Berechnung des Skalarprodukts sowie der Beschleunigung
verringert. Die inverse Wurzel wird durch die \texttt{rsqrt}-Funktion berechnet.
Die Quelltexte~\ref{anhang:hip:bodybodyinteraction} (HIP) und
\ref{anhang:hc:bodybodyinteraction} (HC) im Anhang dieser Arbeit zeigen die
konkrete Implementierung.

Die theoretischen Funktionen~\ref{methoden:nbody:gpu:tilecalculation} und
\ref{methoden:nbody:gpu:calcforces} wurden zusammengefasst, da erstere
nur aus einer kurzen Schleife besteht. Überdies wurde der Compiler angewiesen,
die Schleife auszurollen (siehe auch den nächsten Abschnitt). Diese
Implementierungen finden sich in den angehängten
Quelltexten~\ref{anhang:hip:forcecalculation} (HIP) bzw.
\ref{anhang:hc:forcecalculation} (HC).

\subsubsection{Optimierung und Auswertung}

Eine einfache Optimierung ist das Ausrollen der Schleife, die nacheinander die
Interaktionen berechnet. Dadurch erhöht sich der Registerbedarf pro Thread, der
Overhead, der durch Verzweigungsinstruktionen anfällt, wird jedoch verringert.
Dieser Effekt ist deutlich in den Abbildungen~\ref{amd:nbody:unroll-hc} und
\ref{amd:nbody:unroll-hip} zu sehen:  durch die Bestimmung eines besseren
Ausrollfaktors lassen sich in diesem Benchmark bei einer festen Kachelgröße von
$p = 256$ knapp \num{1000} GFLOPS mehr Durchsatz gewinnen. Anhand dieser Messung
wurde für den weiteren Verlauf der Messungen ein Ausrollfaktor von 8 festgelegt.

\begin{figure}
    \centering
    \begin{tikzpicture}
        \begin{axis}[
            title = {Ausrollen -- HC},
            xlabel = {Ausrollfaktor},
            ylabel = {GFLOPS},
            xmode = log,
            log basis x = 2,
            xmin = 1, xmax = 512,
            xtick = {1,2,4,8,16,32,64,128,256,512},
            log ticks with fixed point,
            ymajorgrids = true,
            xmajorgrids = true,
            grid style = dashed,
            legend cell align = left,
            legend pos = outer north east,
            no markers,
            every axis plot/.append style = {very thick},
            cycle list name = exotic,
            /pgf/number format/.cd, use comma
        ]
            \addplot table [x = count, y = gflops-hc, col sep = semicolon]
                           {data/nbody-amd-unroll-524288.csv};
            \addlegendentry{$n = 524.288$} 

            \addplot table [x = count, y = gflops-hc, col sep = semicolon]
                           {data/nbody-amd-unroll-65536.csv};
            \addlegendentry{$n = 65.536$} 

            \addplot table [x = count, y = gflops-hc, col sep = semicolon]
                           {data/nbody-amd-unroll-8192.csv};
            \addlegendentry{$n = 8.192$} 
        \end{axis}
    \end{tikzpicture}
    \caption{Performanzgewinn durch das Ausrollen der Schleife (HC)}
    \label{amd:nbody:unroll-hc}
\end{figure}

\begin{figure}
    \centering
    \begin{tikzpicture}
        \begin{axis}[
            title = {Ausrollen -- HIP},
            xlabel = {Ausrollfaktor},
            ylabel = {GFLOPS},
            xmode = log,
            log basis x = 2,
            xmin = 1, xmax = 512,
            xtick = {1,2,4,8,16,32,64,128,256,512},
            log ticks with fixed point,
            ymajorgrids = true,
            xmajorgrids = true,
            grid style = dashed,
            legend cell align = left,
            legend pos = outer north east,
            no markers,
            every axis plot/.append style = {very thick},
            cycle list name = exotic,
            /pgf/number format/.cd, use comma
        ]
            \addplot table [x = count, y = gflops-hip, col sep = semicolon]
                           {data/nbody-amd-unroll-524288.csv};
            \addlegendentry{$n = 524.288$} 

            \addplot table [x = count, y = gflops-hip, col sep = semicolon]
                           {data/nbody-amd-unroll-65536.csv};
            \addlegendentry{$n = 65.536$} 

            \addplot table [x = count, y = gflops-hip, col sep = semicolon]
                           {data/nbody-amd-unroll-8192.csv};
            \addlegendentry{$n = 8.192$} 
        \end{axis}
    \end{tikzpicture}
    \caption{Performanzgewinn durch das Ausrollen der Schleife (HIP)}
    \label{amd:nbody:unroll-hip}
\end{figure}

Der nächste performanzrelevante Faktor ist die Größe der Kacheln selbst. Aus den
in den Abbildungen~\ref{amd:nbody:tilesize-hc} und \ref{amd:nbody:tilesize-hip}
dargestellten Messergebnissen wird ersichtlich, dass die Kachelgröße für den
Benchmark weniger wichtig ist; relevante Unterschiede sind nur bei großen
Kachelgrößen und wenigen Elementen sichtbar. Für den weiteren Messverlauf wird
daher eine Kachelgröße von $p = 256$ angenommen.

\begin{figure}
    \centering
    \begin{tikzpicture}
        \begin{axis}[
            title = {Kachelgrößen -- HC},
            xlabel = {Kachelgröße},
            ylabel = {GFLOPS},
            xmode = log,
            log basis x = 2,
            xtick = {64,128,256,512,1024},
            xticklabel = {\xinttheiexpr2^\tick\relax},
            log ticks with fixed point,
            ymajorgrids = true,
            xmajorgrids = true,
            grid style = dashed,
            legend cell align = left,
            legend pos = outer north east,
            no markers,
            every axis plot/.append style = {very thick},
            cycle list name = exotic,
            /pgf/number format/.cd, use comma
        ]
            \addplot table [x = tilesize, y = gflops-hc, col sep = semicolon]
                           {data/nbody-amd-tilesize-524288.csv};
            \addlegendentry{$n = 524.288$} 

            \addplot table [x = tilesize, y = gflops-hc, col sep = semicolon]
                           {data/nbody-amd-tilesize-65536.csv};
            \addlegendentry{$n = 65.536$} 

            \addplot table [x = tilesize, y = gflops-hc, col sep = semicolon]
                           {data/nbody-amd-tilesize-8192.csv};
            \addlegendentry{$n = 8.192$} 
        \end{axis}
    \end{tikzpicture}
    \caption{Performanz bei verschiedenen Kachelgrößen (HC)}
    \label{amd:nbody:tilesize-hc}
\end{figure}

\begin{figure}
    \centering
    \begin{tikzpicture}
        \begin{axis}[
            title = {Kachelgrößen -- HIP},
            xlabel = {Kachelgröße},
            ylabel = {GFLOPS},
            xmode = log,
            log basis x = 2,
            xtick = {64,128,256,512,1024},
            xticklabel = {\xinttheiexpr2^\tick\relax},
            log ticks with fixed point,
            ymajorgrids = true,
            xmajorgrids = true,
            grid style = dashed,
            legend cell align = left,
            legend pos = outer north east,
            no markers,
            every axis plot/.append style = {very thick},
            cycle list name = exotic,
            /pgf/number format/.cd, use comma
        ]
            \addplot table [x = tilesize, y = gflops-hip, col sep = semicolon]
                           {data/nbody-amd-tilesize-524288.csv};
            \addlegendentry{$n = 524.288$} 

            \addplot table [x = tilesize, y = gflops-hip, col sep = semicolon]
                           {data/nbody-amd-tilesize-65536.csv};
            \addlegendentry{$n = 65.536$} 

            \addplot table [x = tilesize, y = gflops-hip, col sep = semicolon]
                           {data/nbody-amd-tilesize-8192.csv};
            \addlegendentry{$n = 8.192$} 
        \end{axis}
    \end{tikzpicture}
    \caption{Performanz bei verschiedenen Kachelgrößen (HIP)}
    \label{amd:nbody:tilesize-hip}
\end{figure}

Mit der experimentell ermittelten Konfiguration lässt sich ein direkter
Vergleich zwischen \gls{hc} und \gls{hip} anstellen. Die
Abbildung~\ref{amd:nbody:comparison} zeigt, dass die Performanz bei beiden
\gls{api}s nahezu identisch ist. Der Blick in den generierten Maschinen-Code
zeigt, dass der \texttt{hcc}-Compiler in der Lage ist, für beide Varianten ein
identisches Ergebnis zu erzeugen (siehe Quelltexte~\ref{amd:nbody:isahc} und
\ref{amd:nbody:isahip}).

\begin{figure}
    \centering
    \begin{tikzpicture}
        \begin{axis}[
            title = {Leistungsvergleich -- HC und HIP},
            xlabel = {$n$},
            ylabel = {GFLOPS},
            xtick = data,
            xmode = log,
            log basis x = 2,
            xticklabel = {\xinttheiexpr2^\tick\relax},
            ymajorgrids = true,
            xmajorgrids = true,
            grid style = dashed,
            legend cell align = left,
            legend pos = outer north east,
            no markers,
            every axis plot/.append style = {very thick},
            cycle list name = exotic,
            /pgf/number format/.cd, use comma,
            ybar,
            width = 0.75\textwidth,
            scale only axis,
            ymin = 0, ymax = 13000,
            extra y ticks = 12583,
            extra y tick labels = {},
            extra y tick style={grid=major, major grid style={solid,thick,draw=red}},
            scaled y ticks = false,
            ylabel near ticks,
            xlabel near ticks
        ]
            \addplot table [x = n, y = gflops-hc, col sep = semicolon]
                           {data/nbody-amd.csv};
            \addlegendentry{HC} 

            \addplot table [x = n, y = gflops-hip, col sep = semicolon]
                           {data/nbody-amd.csv};
            \addlegendentry{HIP} 

            \addlegendimage{peak}
            \addlegendentry{Max.}
        \end{axis}
    \end{tikzpicture}
    \caption{Leistungsvergleich zwischen HC und HIP}
    \label{amd:nbody:comparison}
\end{figure}

\begin{figure}
    \begin{minipage}{0.5\textwidth}
        \centering
        \begin{minted}[fontsize=\footnotesize]{text}
ds_read2_b64 v[14:17], v10 offset1:1
v_add_u32_e32 v9, 64, v9
s_waitcnt lgkmcnt(0)
v_sub_f32_e32 v16, v16, v5
v_sub_f32_e32 v15, v15, v4
v_fma_f32 v18, v16, v16, s20
v_sub_f32_e32 v14, v14, v3
v_fma_f32 v18, v15, v15, v18
v_fma_f32 v18, v14, v14, v18
v_mul_f32_e32 v19, v18, v18
v_mul_f32_e32 v18, v18, v19
v_cmp_gt_f32_e32 vcc, s21, v18
v_mov_b32_e32 v19, s22
v_cndmask_b32_e32 v20, 1.0, v19, vcc
v_mul_f32_e32 v18, v18, v20
v_rsq_f32_e32 v18, v18
v_mov_b32_e32 v20, s23
v_cndmask_b32_e32 v21, 1.0, v20, vcc
v_mul_f32_e32 v18, v21, v18
v_mul_f32_e32 v17, v17, v18
v_fma_f32 v18, v14, v17, v11
v_fma_f32 v15, v15, v17, v12
v_fma_f32 v16, v16, v17, v13                   
        \end{minted}
        \captionof{listing}{Maschinencode des HC-Kernels}
        \label{amd:nbody:isahc}
    \end{minipage}
    %
    \begin{minipage}{0.5\textwidth}
        \centering
        \begin{minted}[fontsize=\footnotesize]{text}
ds_read2_b64 v[14:17], v9 offset1:1
v_add_u32_e32 v10, 64, v10
s_waitcnt lgkmcnt(0)
v_sub_f32_e32 v16, v16, v5
v_sub_f32_e32 v15, v15, v4
v_fma_f32 v18, v16, v16, s16
v_sub_f32_e32 v14, v14, v3
v_fma_f32 v18, v15, v15, v18
v_fma_f32 v18, v14, v14, v18
v_mul_f32_e32 v19, v18, v18
v_mul_f32_e32 v18, v18, v19
v_cmp_gt_f32_e32 vcc, s17, v18
v_mov_b32_e32 v19, s18
v_cndmask_b32_e32 v20, 1.0, v19, vcc
v_mul_f32_e32 v18, v18, v20
v_rsq_f32_e32 v18, v18
v_mov_b32_e32 v20, s19
v_cndmask_b32_e32 v21, 1.0, v20, vcc
v_mul_f32_e32 v18, v21, v18
v_mul_f32_e32 v17, v17, v18
v_fma_f32 v18, v14, v17, v11
v_fma_f32 v15, v15, v17, v12
v_fma_f32 v16, v16, v17, v13
        \end{minted}
        \captionof{listing}{Maschinencode des HIP-Kernels}
        \label{amd:nbody:isahip}
    \end{minipage}
\end{figure}

Beide \gls{api}s erreichen jedoch nicht einmal die Hälfte der theoretisch
möglichen FLOPS.
