\appendix

\section{CUDA-Kernel}

\subsection{zcopy}

\begin{code}
    \begin{minted}[breaklines,breakafter=\,,escapeinside=||,fontsize=\small]{cuda}
__global__ void read_write(const float4* __restrict__ A,
                                 float4* __restrict__ B,
                           std::size_t num_elems)
{
    auto stride = gridDim.x * blockDim.x;
    for(auto i = blockIdx.x * blockDim.x + threadIdx.x;
             i < elems;
             i += stride)
    {
        B[i] = A[i];
    }
}

__global__ void write(float4* __restrict__ B, std::size_t num_elems)
{
    auto stride = gridDim.x * blockDim.x;
    for(auto i = blockIdx.x * blockDim.x + threadIdx.x;
             i < elems;
             i += stride)
    {
        B[i] = make_float4{0.f, 0.f, 0.f, 0.f};
    }
}
    \end{minted}
    \caption{zcopy -- CUDA-Implementierung}
    \label{anhang:cuda:zcopy}
\end{code}

\subsection{Reduce}

\subsection{N-Body}

\section{HIP-Kernel}

\subsection{zcopy}

\begin{code}
    \begin{minted}[breaklines,breakafter=\,,escapeinside=||,fontsize=\small]{cuda}
__global__ void read_write(const float4* __restrict__ A,
                                 float4* __restrict__ B,
                           std::size_t num_elems)
{
    auto stride = |\textcolor{keyword-green}{hipGridDim\_x}| * |\textcolor{keyword-green}{hipBlockDim\_x}|;
        for(auto i = |\textcolor{keyword-green}{hipBlockIdx\_x}| * |\textcolor{keyword-green}{hipBlockDim\_x}| + |\textcolor{keyword-green}{hipThreadIdx\_x}|;
             i < elems;
             i += stride)
    {
        B[i] = A[i];
    }
}

__global__ void write(float4* __restrict__ B, std::size_t num_elems);
{
    auto stride = |\textcolor{keyword-green}{hipGridDim\_x}| * |\textcolor{keyword-green}{hipBlockDim\_x}|;
    for(auto i = |\textcolor{keyword-green}{hipBlockIdx\_x}| * |\textcolor{keyword-green}{hipBlockDim\_x}| + |\textcolor{keyword-green}{hipThreadIdx\_x}|;
             i < elems;
             i += stride)
    {
        B[i] = make_float4{0.f, 0.f, 0.f, 0.f};
    }
}
    \end{minted}
    \caption{zcopy -- HIP-Implementierung}
    \label{anhang:hip:zcopy}
\end{code}

\subsection{Reduce}
\subsection{N-Body}

\begin{code}
    \begin{minted}[breaklines,breakafter=\,,escapeinside=||,fontsize=\small]{cuda}
|\textbf{\textcolor{keyword-green}{constexpr}}| auto eps = 0.001f;
|\textbf{\textcolor{keyword-green}{constexpr}}| auto eps2 = eps * eps;

__device__ auto body_body_interaction(float4 bi, float4 bj, float3 ai)
-> float3
{
    // r_ij [3 FLOPS]
    auto r = float3{};
    r.x = bj.x - bi.x;
    r.y = bj.y - bi.y;
    r.z = bj.z - bi.z;

    // dist_sqr = skalarprodukt(r_ij, r_ij) + epsilon^2 [6 FLOPS]
    auto dist_sqr = fmaf(r.x, r.x,
                         fmaf(r.y, r.y,
                              fmaf(r.z, r.z, eps2)));

    // inv_dist_cube = 1 / dist_sqr^(3/2) [4 FLOPS]
    auto dist_sixth = dist_sqr * dist_sqr * dist_sqr;
    auto inv_dist_cube = rsqrtf(dist_sixth);

    // s = m_j * inv_dist_cube [1 FLOP]
    auto s = bj.w * inv_dist_cube;

    // a_i = a_i + s * r_ij [6 FLOPS]
    ai.x = fmaf(r.x, s, ai.x);
    ai.y = fmaf(r.y, s, ai.y);
    ai.z = fmaf(r.z, s, ai.z);

    return ai;
}
    \end{minted}
    \caption{body\_body\_interaction - HIP-Implementierung}
    \label{anhang:hip:bodybodyinteraction}
\end{code}

\begin{code}
    \begin{minted}[breaklines,breakafter=\,,escapeinside=||,fontsize=\small]{cuda}
__device__ auto force_calculation(float4 body_pos,
                                  const float4* positions,
                                  unsigned tiles)
-> float3
{
    extern __shared__ float4 sh_position[];

    auto acc = float3{};

    for(auto tile = 0u; tile < tiles; ++tile)
    {
        auto idx = tile * |\textcolor{keyword-green}{hipBlockDim\_x}| + |\textcolor{keyword-green}{hipThreadIdx\_x}|;

        sh_position[|\textcolor{keyword-green}{hipThreadIdx\_x}|] = positions[idx];
        __syncthreads();

        // entspricht tile_calculation()
        #pragma unroll
        for(auto i = 0u; i < |\textcolor{keyword-green}{hipBlockDim\_x}|; ++i)
            acc = body_body_interaction(body_pos, sh_position[i],
                                        acc);
        __syncthreads();
    }
    return acc;
}
    \end{minted}
    \caption{force\_calculation - HIP-Implementierung}
    \label{anhang:hip:forcecalculation}
\end{code}


\section{HC-Kernel}

\subsection{zcopy}

\begin{code}
    \begin{minted}[breaklines,breakafter=\,,escapeinside=||,fontsize=\small]{c++}
[[hc]]
template <typename DataT>
void read_write(hc::tiled_index<1> idx, hc::array_view<DataT, 1> A,
                hc::array_view<DataT, 1> B, std::size_t elems,
                unsigned int tiles, unsigned int tile_size)
{
    auto stride = tiles * tile_size;
    for(auto i = idx.tile[0] * tile_size + idx.local[0];
             i < elems;
             i += stride)
    {
        B[i] = A[i];
    }
}

[[hc]]
template <typename DataT>
void write(hc::tiled_index<1> idx, hc::array_view<DataT, 1> B,
           std::size_t elems, unsigned int tiles,
           unsigned int tile_size)
{
    auto stride = tiles * tile_size;
    for(auto i = idx.tile[0] * tile_size + idx.local[0];
             i < elems;
             i += stride)
    {
        B[i] = DataT{};
    }
}
    \end{minted}
    \caption{zcopy - HC-Implementierung}
    \label{anhang:hc:zcopy}
\end{code}

\subsection{Reduce}
\subsection{N-Body}

\begin{code}
    \begin{minted}[breaklines,breakafter=\,,escapeinside=||,fontsize=\small]{c++}
using float3 = hc::short_vector::float3;
using float4 = hc::short_vector::float4;

[[hc]]
auto body_body_interaction(float4 bi, float4 bj, float3 ai) -> float3
{
    constexpr auto eps = 0.001f;
    constexpr auto eps2 = eps * eps;

    // r_ij [3 FLOPS]
    auto r = bj.get_xyz() - bi.get_xyz();

    // dist_sqr = skalarprodukt(r, r) + epsilon^2 [6 FLOPS]
    auto dist_sqr = fmaf(r.x, r.x,
                         fmaf(r.y, r.y,
                              fmaf(r.z, r.z, eps2)));

    // inv_dist_cube = 1 / dist_sqr^(3/2) [4 FLOPS]
    auto dist_sixth = dist_sqr * dist_sqr * dist_sqr;
    auto inv_dist_cube = rsqrtf(dist_sixth);

    // s = m_j * inv_dist_cube [1 FLOP]

    // a_i = a_i + s * r_ij [6 FLOPS]
    ai.x = fmaf(r.x, s, ai.x);
    ai.y = fmaf(r.y, s, ai.y);
    ai.z = fmaf(r.z, s, ai.z);

    return ai;
}
    \end{minted}
    \caption{body\_body\_interaction - HC-Implementierung}
    \label{anhang:hc:bodybodyinteraction}
\end{code}

\begin{code}
    \begin{minted}[breaklines,breakafter=\,,escapeinside=||,fontsize=\small]{c++}
using float3 = hc::short_vector::float3;
using float4 = hc::short_vector::float4;

[[hc]]
auto force_calculation(hc::tiled_index<1> idx, float4 body_pos,
                       hc::array_view<const float4, 1> positions,
                       float4* sh_position, unsigned tiles)
-> float3
{
    auto acc = float3{};

    for(auto tile = 0u; tile < tiles; ++tile)
    {
        auto id = tile * idx.tile_dim[0] + idx.local[0];

        sh_position[idx.local[0]] = positions[id];
        idx.barrier.wait_with_tile_static_memory_fence();

        // entspricht tile_calculation()
        #pragma unroll
        for(auto i = 0u; i < idx.tile_dim[0]; ++i)
            acc = body_body_interaction(body_pos, sh_position[i],
                                        acc);
        idx.barrier.wait_with_tile_static_memory_fence();
    }
    return acc;
}
    \end{minted}
    \caption{force\_calculation - HC-Implementierung}
    \label{anhang:hc:forcecalculation}
\end{code}

\section{SYCL-Kernel}

\subsection{zcopy}

\begin{code}
    \begin{minted}[breaklines,breakafter=\,,escapeinside=||,fontsize=\small]{c++}
struct reader_writer
{
    cl::sycl::accessor<cl::sycl::float4, 1,
                       cl::sycl::access::mode::read,
                       cl::sycl::access::target::global_buffer> A;
    cl::sycl::accessor<cl::sycl::float4, 1,
                       cl::sycl::access::mode::discard_write,
                       cl::sycl::access::target::global_buffer> B;

    auto operator()(cl::sycl::nd_item<1> my_item) -> void
    {
        auto stride = my_item.get_group_range(0) *
                      my_item.get_local_range(0);
        for(auto i = my_item.get_global_id(0); i < elems; i += stride)
        {
            B[i] = A[i];
        }
    }
};

struct writer
{
    cl::sycl::accessor<cl::sycl::float4, 1,
                       cl::sycl::access::mode::discard_write,
                       cl::sycl::access::target::global_buffer> B;

    auto operator()(cl::sycl::nd_item<1> my_item) -> void
    {
        auto stride = my_item.get_group_range(0) *
                      my_item.get_local_range(0);
        for(auto i = my_item.get_global_id(0); i < elems; i += stride)
        {
            B[i] = cl::sycl::float4{};
        }
    }
};
    \end{minted}
    \caption{zcopy -- SYCL-Implementierung}
    \label{anhang:sycl:zcopy}
\end{code}

\subsection{Reduce}
\subsection{N-Body}
