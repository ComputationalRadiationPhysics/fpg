\section{Messungen auf NVIDIA-GPUs}
\label{nvidia}

\subsection{Verwendete Hard- und Software}

Die hier gezeigten Benchmark-Ergebnisse wurden auf verschiedenen Knoten des
\gls{hpc}-Systems Taurus gemessen:

\begin{itemize}
    \item Ein Knoten der \texttt{gpu2}-Partition mit zwei GPUs des Modells Tesla
          K20x (mit aktiviertem ECC).
    \item Einem experimentellen Knoten mit je einer GPU der Modelle Tesla P100
          und Tesla V100
\end{itemize}

Die Messungen fanden innerhalb der SCS5-Umgebung statt. Das verwendete
CUDA-Modul war \texttt{CUDA/10.0.130}, der zugehörige Host-Compiler
\texttt{GCC/7.3.0-2.30}. HIP wurde aus dem GitHub-Repository des ROCm-Projektes
in der Version 1.5.19061 heruntergeladen und lokal installiert. Als
SYCL-Implementierung kam ComputeCpp der Firma Codeplay in der Version 1.0.5 für
CentOS zum Einsatz, welches ebenfalls lokal installiert wurde.

\subsection{zcopy}

\subsubsection{Vorüberlegungen}

Um eine optimale Ausnutzung der Speicherbandbreite zu erreichen, ist ein genauer
Blick auf die verwendete Architektur erforderlich. Die K20x-GPUs weisen hier
größere Unterschiede zu den P100- und V100-Modellen auf, die sich auf den
betrachteten Benchmark auswirken können. Während die P100/V100-GPUs auf jedem
Multiprozessor bei einfacher Genauigkeit \num{64} Hardware-Threads zur Verfügung
haben, sind es bei den K20x-GPUs \num{192}. Ein Multiprozessor der
P100/V100-GPUs kann somit zwei \textit{Warps} parallel ausführen, während auf
der K20x-GPU sechs parallele \textit{Warps} pro Multiprozessor möglich sind. Die
unterschiedliche \textit{Warp}-Zahl wirkt sich unter Umständen auf die
Bandbreite des Speichers aus, da jeder K20x-Multiprozessor mehr Zugriffe
benötigt als seine P100/V100-Gegenstücke. Aus diesem Grund werden auf der
K20x-GPU und den P100/V100-GPUs Blockgrößen untersucht, die ein Vielfaches von
\num{64} sind (bis hin zu \num{1024}). Zusätzlich werden auf der K20x
Blockgrößen betrachtet, die ein Vielfaches von \num{192} sind (bis hin zu
\num{768}), um eventuelle Effekte zu entdecken.

Alle verwendeten NVIDIA-GPUs besitzen eine L1-Cacheline-Größe von \num{128}
Byte. Speicherzugriffe der \textit{Warps} werden auf \textit{Half-Warps} (also
\num{16} Threads des \textit{Warps}) aufgeteilt. Sofern alle Threads eines
\textit{Half Warps} benachbarte \num{8}-Byte-Sequenzen laden, lassen sich diese
Zugriffe in einer einzigen Cacheline vereinen, was die pro Thread verwendete
Speicherbandbreite drastisch reduziert und eine insgesamt höhere Auslastung
erlaubt.

Um eine höhere Auslastung des Speicher-Controllers zu erreichen und die Anzahl
der notwendigen Schleifendurchläufe innerhalb des Kernels zu verringern, wodurch
sich die Zahl abhängiger Instruktionen verringert, werden die \num{8} Byte pro
Thread in diesem Benchmark verdoppelt. Jeder Thread lädt also \num{16} Byte in
Form eines Elements vom Typ \texttt{float4}. Die Anhang befindlichen
Quelltexte~\ref{anhang:cuda:zcopy} (CUDA), \ref{anhang:hip:zcopy} (\gls{hip})
und \ref{anhang:sycl:zcopy} (SYCL) zeigen die Kernel-Implementierungen der
verschiedenen Spracherweiterungen.

\subsubsection{Ergebnisse}

Ein Blick auf die mit CUDA ermittelten Ergebnisse zeigt, dass die K20x-GPU bei
Speicherzugriffen von größeren Thread-Blöcken als auch von vielen Thread-Blöcken
profitiert. Dies gilt sowohl für die \glqq generischen\grqq\ Blockgrößen, die
ein Vielfaches von \num{64} darstellen (siehe 
Abbildung~\ref{nvidia:zcopy:k20xgenerisch}, als auch die auf die
Kepler-Architektur angepassten Blockgrößen (siehe
Abbildung~\ref{nvidia:zcopy:k20xangepasst}). Außerdem bestätigt sich die
Annahme, dass mit der angepassten Blockgröße eine bessere Bandbreite erzielt
werden kann (siehe Abbildung~\ref{nvidia:zcopy:k20xvergleich}). Dies hilft
insbesondere dem kombinierten Lese- und Schreibvorgang, der dadurch nahezu die
selbe Bandbreite wie der reine Schreibvorgang erreicht (siehe
Abbildung~\ref{nvidia:zcopy:k20xrwwo}).

Die K20x-GPU erreicht etwa 75\% der theoretisch möglichen Bandbreite (siehe
Abbildung~\ref{nvidia:zcopy:k20xmax}). Dies ist durch das aktivierte ECC und
die auf dem Taurus-System nicht steuerbaren Taktraten der GPU zu erklären.

\subsection{Reduction}

\subsection{N-Body}

Ein gesonderter Vergleich ist zwischen CUDA und SYCL nötig, da das
experimentelle Backend für NVIDIA-GPUs der ComputeCpp-Implementierung die
Funktion \texttt{rsqrtf} für die reziproke Quadratwurzel nicht unterstützt. Da
die äquivalente Berechnung \texttt{1.f / sqrtf} deutlich langsamer ist, wurde
stattdessen eine schnellere, weniger genaue Implementierung der Funktion
\texttt{rsqrtf} aus dem Quelltext des Spiels \textit{Quake 3 Arena} übernommen
(siehe Quelltext~\ref{nvidia:nbody:qrsqrt} für eine normale
C++-Implementierung).

\begin{code}
    \begin{minted}[fontsize=\small]{c++}
auto Q_rsqrt(float number) -> float
{
    auto x2 = number * 0.5f;
    auto y = number;
    auto i = *(reinterpret_cast<std::int32_t*>(&y));

    i = 0x5f3759df - (i >> 1);

    y = *(reinterpret_cast<float*>(&i));
    y *= 1.5f - (x2 * y * y);

    return y;
}
    \end{minted}
    \caption{Quake-3-Implementierung der rsqrt-Funktion}
    \label{nvidia:nbody:qrsqrt}
\end{code}
