\section{Messungen auf NVIDIA-GPUs}
\label{nvidia}

\subsection{Verwendete Hard- und Software}

Die hier gezeigten Benchmark-Ergebnisse wurden auf verschiedenen Knoten des
\gls{hpc}-Systems Taurus gemessen:

\begin{itemize}
    \item Ein Knoten der \texttt{gpu2}-Partition mit zwei GPUs des Modells Tesla
          K20x (mit aktiviertem ECC).
    \item Einem experimentellen Knoten mit je einer GPU der Modelle Tesla P100
          und Tesla V100
\end{itemize}

Die Messungen fanden innerhalb der SCS5-Umgebung statt, in der die folgende
Software verwendet wurde:

\begin{itemize}
    \item \texttt{CUDA/10.0.130} (Modulsystem)
    \item \texttt{GCC/7.3.0-2.30} (Modulsystem)
    \item \gls{hip} (Version 1.5.19061, ROCm-GitHub-Repository)
    \item ComputeCpp (Codeplays SYCL-Implementierung, Version 1.0.5 für Ubuntu
          14.04)
\end{itemize}

\subsection{zcopy}

\subsubsection{Vorüberlegungen}

Um eine optimale Ausnutzung der Speicherbandbreite zu erreichen, ist ein genauer
Blick auf die verwendete Architektur erforderlich. Die K20x-GPUs weisen hier
größere Unterschiede zu den P100- und V100-Modellen auf, die sich auf den
betrachteten Benchmark auswirken können. Während die P100/V100-GPUs auf jedem
Multiprozessor bei einfacher Genauigkeit \num{64} Hardware-Threads zur Verfügung
haben, sind es bei den K20x-GPUs \num{192}. Ein Multiprozessor der
P100/V100-GPUs kann somit zwei \textit{Warps} parallel ausführen, während auf
der K20x-GPU sechs parallele \textit{Warps} pro Multiprozessor möglich sind. Die
unterschiedliche \textit{Warp}-Zahl wirkt sich unter Umständen auf die
Bandbreite des Speichers aus, da jeder K20x-Multiprozessor mehr Zugriffe
benötigt als seine P100/V100-Gegenstücke. Aus diesem Grund werden auf der
K20x-GPU und den P100/V100-GPUs Blockgrößen untersucht, die ein Vielfaches von
\num{64} sind (bis \num{1024}). Zusätzlich werden auf der K20x Blockgrößen
betrachtet, die ein Vielfaches von \num{192} sind (bis \num{768}), um eventuelle
Effekte zu entdecken.

Alle verwendeten NVIDIA-GPUs besitzen eine L1-Cacheline-Größe von \num{128}
Byte. Speicherzugriffe der \textit{Warps} werden auf \textit{Half-Warps} (also
\num{16} Threads des \textit{Warps}) aufgeteilt. Sofern alle Threads eines
\textit{Half Warps} benachbarte \num{8}-Byte-Sequenzen laden, lassen sich diese
Zugriffe in einer einzigen Cacheline vereinen, was die pro Thread verwendete
Speicherbandbreite drastisch reduziert und eine insgesamt höhere Auslastung
erlaubt.

Um eine höhere Auslastung des Speicher-Controllers zu erreichen und die Anzahl
der notwendigen Schleifendurchläufe innerhalb des Kernels zu verringern, wodurch
sich die Zahl abhängiger Instruktionen verringert, werden die \num{8} Byte pro
Thread in diesem Benchmark verdoppelt. Jeder Thread lädt also \num{16} Byte in
Form eines Elements vom Typ \texttt{float4}. Die Anhang befindlichen
Quelltexte~\ref{anhang:cuda:zcopy} (CUDA), \ref{anhang:hip:zcopy} (\gls{hip})
und \ref{anhang:sycl:zcopy} (SYCL) zeigen die Kernel-Implementierungen der
verschiedenen Spracherweiterungen.

\subsubsection{Messmethoden}

Der Speicher der GPU wurde zunächst mit zwei gleich großen Datenfeldern $A$ und
$B$ befüllt, die jeweils etwas weniger als die Hälfte des GPU-Speichers
einnahmen und $n$ \texttt{float4}-Elemente umfassten. Diese wurden mit Nullen
($A$) bzw.\ \textit{NaN} ($B$) initialisiert. Anschließend wurden die Kernel für
jede Block-Größe jeweils zehn Mal ausgeführt, wobei für jeden Kernel-Durchlauf
die benötigte Zeit über die Event-Funktionalitäten der verschiedenen
Spracherweiterungen gemessen wurde. Der minimale Zeitbedarf $t_{\text{min}}$
(in \si{\second}) jedes Kernels wird als Grundlage für die Berechnung der
Bandbreite $B$ (in \si{\gibi\byte\per\second}) genommen:

\[
    B = \frac{k \cdot \text{sizeof(\texttt{float4})} \cdot n}{t_\text{min}},
\]

wobei der Faktor $k$ die Werte $2 \cdot 10^{-9}$ (kombinierter Lese- und
Schreibzugriff) oder $1 \cdot 10^{-9}$ (Schreibzugriff) annehmen kann.

\subsubsection{Ergebnisse}

Ein Blick auf die mit CUDA ermittelten Ergebnisse zeigt, dass die K20x-GPU bei
Speicherzugriffen von größeren Thread-Blöcken als auch von vielen Thread-Blöcken
profitiert. Dies gilt sowohl für die \glqq generischen\grqq\ Blockgrößen, die
ein Vielfaches von \num{64} darstellen (siehe die
Abbildungen~\ref{nvidia:zcopy:k20xgenerischrw} und
\ref{nvidia:zcopy:k20xgenerischw}), als auch die auf die Kepler-Architektur
angepassten Blockgrößen (siehe die
Abbildungen~\ref{nvidia:zcopy:k20xangepasstrw} und
\ref{nvidia:zcopy:k20xangepasstw}). Außerdem bestätigt sich die Annahme, dass
mit der angepassten Blockgröße eine etwas bessere Bandbreite erzielt werden kann
(siehe Abbildung~\ref{nvidia:zcopy:k20xvergleich}). Dies hilft insbesondere dem
kombinierten Lese- und Schreibvorgang, der dadurch bei einer großen Blockzahl
nahezu die selbe Bandbreite wie der reine Schreibvorgang erreicht (siehe
Abbildung~\ref{nvidia:zcopy:k20xrwwo}).

\begin{figure}
    \centering
    \begin{tikzpicture}
        \begin{axis}[
            title = {zcopy -- Lesen + Schreiben -- generisch -- K20x},
            xlabel = {Blöcke pro Multiprozessor},
            ylabel = {Bandbreite [\si{\gibi\byte\per\second}]},
            xmode = log,
            log basis x = 2,
            xmin = 1, xmax = 8192,
            ymin = 40, ymax = 190,
            xticklabel = {\xinttheiexpr2^\tick\relax},
            log ticks with fixed point,
            ymajorgrids = true,
            xmajorgrids = true,
            grid style = dashed,
            legend cell align = left,
            legend pos = south east,
            no markers,
            every axis plot/.append style = {very thick},
            width = 0.75\textwidth,
            scale only axis,
            cycle list name = exotic,
            /pgf/number format/.cd, use comma
        ]
            \addplot table [x = blocks_per_sm, y = throughput-64,
                            col sep = semicolon]
                           {data/zcopy-nvidia-cuda-k20x-generic-rw.csv};
            \addlegendentry{$\text{Blockgröße} = 64$} 

            \addplot table [x = blocks_per_sm, y = throughput-128,
                            col sep = semicolon]
                           {data/zcopy-nvidia-cuda-k20x-generic-rw.csv};
            \addlegendentry{$\text{Blockgröße} = 128$} 

            \addplot table [x = blocks_per_sm, y = throughput-256,
                            col sep = semicolon]
                           {data/zcopy-nvidia-cuda-k20x-generic-rw.csv};
            \addlegendentry{$\text{Blockgröße} = 256$} 

            \addplot table [x = blocks_per_sm, y = throughput-512,
                            col sep = semicolon]
                           {data/zcopy-nvidia-cuda-k20x-generic-rw.csv};
            \addlegendentry{$\text{Blockgröße} = 512$} 

            \addplot table [x = blocks_per_sm, y = throughput-1024,
                            col sep = semicolon]
                           {data/zcopy-nvidia-cuda-k20x-generic-rw.csv};
            \addlegendentry{$\text{Blockgröße} = 1024$} 
        \end{axis}
    \end{tikzpicture}
    \caption{K20x-Bandbreite für generische Blöcke (Lesen und Schreiben, CUDA)}
    \label{nvidia:zcopy:k20xgenerischrw}
\end{figure}

\begin{figure}
    \centering
    \begin{tikzpicture}
        \begin{axis}[
            title = {zcopy -- Schreiben -- generisch -- K20x},
            xlabel = {Blöcke pro Multiprozessor},
            ylabel = {Bandbreite [\si{\gibi\byte\per\second}]},
            xmode = log,
            log basis x = 2,
            xmin = 1, xmax = 8192,
            ymin = 70, ymax = 190,
            xticklabel = {\xinttheiexpr2^\tick\relax},
            log ticks with fixed point,
            ymajorgrids = true,
            xmajorgrids = true,
            grid style = dashed,
            legend cell align = left,
            legend pos = south east,
            no markers,
            every axis plot/.append style = {very thick},
            width = 0.75\textwidth,
            scale only axis,
            cycle list name = exotic,
            /pgf/number format/.cd, use comma
        ]
            \addplot table [x = blocks_per_sm, y = throughput-64,
                            col sep = semicolon]
                           {data/zcopy-nvidia-cuda-k20x-generic-w.csv};
            \addlegendentry{$\text{Blockgröße} = 64$} 

            \addplot table [x = blocks_per_sm, y = throughput-128,
                            col sep = semicolon]
                           {data/zcopy-nvidia-cuda-k20x-generic-w.csv};
            \addlegendentry{$\text{Blockgröße} = 128$} 

            \addplot table [x = blocks_per_sm, y = throughput-256,
                            col sep = semicolon]
                           {data/zcopy-nvidia-cuda-k20x-generic-w.csv};
            \addlegendentry{$\text{Blockgröße} = 256$} 

            \addplot table [x = blocks_per_sm, y = throughput-512,
                            col sep = semicolon]
                           {data/zcopy-nvidia-cuda-k20x-generic-w.csv};
            \addlegendentry{$\text{Blockgröße} = 512$} 

            \addplot table [x = blocks_per_sm, y = throughput-1024,
                            col sep = semicolon]
                           {data/zcopy-nvidia-cuda-k20x-generic-w.csv};
            \addlegendentry{$\text{Blockgröße} = 1024$} 
        \end{axis}
    \end{tikzpicture}
    \caption{K20x-Bandbreite für generische Blöcke (Schreiben, CUDA)}
    \label{nvidia:zcopy:k20xgenerischw}
\end{figure}

\begin{figure}
    \centering
    \begin{tikzpicture}
        \begin{axis}[
            title = {zcopy -- Lesen + Schreiben -- optimiert -- K20x},
            xlabel = {Blöcke pro Multiprozessor},
            ylabel = {Bandbreite [\si{\gibi\byte\per\second}]},
            xmode = log,
            log basis x = 2,
            xmin = 1, xmax = 8192,
            ymin = 40, ymax = 190,
            xticklabel = {\xinttheiexpr2^\tick\relax},
            log ticks with fixed point,
            ymajorgrids = true,
            xmajorgrids = true,
            grid style = dashed,
            legend cell align = left,
            legend pos = south east,
            no markers,
            every axis plot/.append style = {very thick},
            width = 0.75\textwidth,
            scale only axis,
            cycle list name = exotic,
            /pgf/number format/.cd, use comma
        ]
            \addplot table [x = blocks_per_sm, y = throughput-192,
                            col sep = semicolon]
                           {data/zcopy-nvidia-cuda-k20x-optimized-rw.csv};
            \addlegendentry{$\text{Blockgröße} = 192$} 

            \addplot table [x = blocks_per_sm, y = throughput-384,
                            col sep = semicolon]
                           {data/zcopy-nvidia-cuda-k20x-optimized-rw.csv};
            \addlegendentry{$\text{Blockgröße} = 384$} 

            \addplot table [x = blocks_per_sm, y = throughput-768,
                            col sep = semicolon]
                           {data/zcopy-nvidia-cuda-k20x-optimized-rw.csv};
            \addlegendentry{$\text{Blockgröße} = 768$} 

        \end{axis}
    \end{tikzpicture}
    \caption{K20x-Bandbreite für optimierte Blöcke (Lesen und Schreiben, CUDA)}
    \label{nvidia:zcopy:k20xangepasstrw}
\end{figure}

\begin{figure}
    \centering
    \begin{tikzpicture}
        \begin{axis}[
            title = {zcopy -- Schreiben -- optimiert -- K20x},
            xlabel = {Blöcke pro Multiprozessor},
            ylabel = {Bandbreite [\si{\gibi\byte\per\second}]},
            xmode = log,
            log basis x = 2,
            xmin = 1, xmax = 8192,
            ymin = 70, ymax = 190,
            xticklabel = {\xinttheiexpr2^\tick\relax},
            log ticks with fixed point,
            ymajorgrids = true,
            xmajorgrids = true,
            grid style = dashed,
            legend cell align = left,
            legend pos = south east,
            no markers,
            every axis plot/.append style = {very thick},
            width = 0.75\textwidth,
            scale only axis,
            cycle list name = exotic,
            /pgf/number format/.cd, use comma
        ]
            \addplot table [x = blocks_per_sm, y = throughput-192,
                            col sep = semicolon]
                           {data/zcopy-nvidia-cuda-k20x-optimized-w.csv};
            \addlegendentry{$\text{Blockgröße} = 192$} 

            \addplot table [x = blocks_per_sm, y = throughput-384,
                            col sep = semicolon]
                           {data/zcopy-nvidia-cuda-k20x-optimized-w.csv};
            \addlegendentry{$\text{Blockgröße} = 384$} 

            \addplot table [x = blocks_per_sm, y = throughput-768,
                            col sep = semicolon]
                           {data/zcopy-nvidia-cuda-k20x-optimized-w.csv};
            \addlegendentry{$\text{Blockgröße} = 768$} 
        \end{axis}
    \end{tikzpicture}
    \caption{K20x-Bandbreite für optimierte Blöcke (Schreiben, CUDA)}
    \label{nvidia:zcopy:k20xangepasstw}
\end{figure}

\begin{figure}
    \centering
    \begin{tikzpicture}
        \begin{axis}[
            title = {zcopy -- Schreiben -- generisch und optimiert -- K20x},
            xlabel = {Blöcke pro Multiprozessor},
            ylabel = {Bandbreite [\si{\gibi\byte\per\second}]},
            xmode = log,
            log basis x = 2,
            xmin = 1, xmax = 8192,
            ymin = 175, ymax = 188,
            xticklabel = {\xinttheiexpr2^\tick\relax},
            log ticks with fixed point,
            ymajorgrids = true,
            xmajorgrids = true,
            grid style = dashed,
            legend cell align = left,
            legend pos = south east,
            no markers,
            every axis plot/.append style = {very thick},
            width = 0.75\textwidth,
            scale only axis,
            cycle list name = exotic,
            /pgf/number format/.cd, use comma
        ]
            \addplot table [x = blocks_per_sm, y = throughput-768,
                            col sep = semicolon]
                           {data/zcopy-nvidia-cuda-k20x-optimized-w.csv};
            \addlegendentry{$\text{Blockgröße} = 768$} 

            \addplot table [x = blocks_per_sm, y = throughput-1024,
                            col sep = semicolon]
                           {data/zcopy-nvidia-cuda-k20x-generic-w.csv};
            \addlegendentry{$\text{Blockgröße} = 1024$} 
        \end{axis}
    \end{tikzpicture}
    \caption{K20x-Bandbreite für generische und optimierte Blöcke (Schreiben,
             CUDA)}
    \label{nvidia:zcopy:k20xvergleich}
\end{figure}

\begin{figure}
    \centering
    \begin{tikzpicture}
        \begin{axis}[
            title = {zcopy -- optimiert -- K20x},
            xlabel = {Blöcke pro Multiprozessor},
            ylabel = {Bandbreite [\si{\gibi\byte\per\second}]},
            xmode = log,
            log basis x = 2,
            xmin = 1, xmax = 8192,
            % ymin = 175, ymax = 188,
            xticklabel = {\xinttheiexpr2^\tick\relax},
            log ticks with fixed point,
            ymajorgrids = true,
            xmajorgrids = true,
            grid style = dashed,
            legend cell align = left,
            legend pos = south east,
            no markers,
            every axis plot/.append style = {very thick},
            width = 0.75\textwidth,
            scale only axis,
            cycle list name = exotic,
            /pgf/number format/.cd, use comma
        ]
            \addplot table [x = blocks_per_sm, y = throughput-768,
                            col sep = semicolon]
                           {data/zcopy-nvidia-cuda-k20x-optimized-rw.csv};
            \addlegendentry{Lesen + Schreiben} 

            \addplot table [x = blocks_per_sm, y = throughput-768,
                            col sep = semicolon]
                           {data/zcopy-nvidia-cuda-k20x-optimized-w.csv};
            \addlegendentry{Schreiben} 
        \end{axis}
    \end{tikzpicture}
    \caption{K20x-Bandbreite für 768er-Blöcke (CUDA)}
    \label{nvidia:zcopy:k20xrwwo}
\end{figure}

Die mit \gls{hip} und SYCL ermittelten Ergebnisse zeigen ein ähnliches Bild
(siehe die angehängten Abbildungen in den
Abschnitten~\ref{anhang:hip:nvzcopyfig} (\gls{hip}) und
\ref{anhang:sycl:zcopyfig} (SYCL)).

Hinsichtlich der erreichten Speicherbandbreite bestehen zwischen den
Spracherweiterungen keine nennenswerten Unterschiede, wie die
Abbildung~\ref{nvidia:zcopy:k20xexts} zeigt.

\begin{figure}
    \centering
    \begin{tikzpicture}
        \begin{axis}[
            title = {zcopy -- Schreiben -- optimiert -- K20x},
            xlabel = {Blöcke pro Multiprozessor},
            ylabel = {Bandbreite [\si{\gibi\byte\per\second}]},
            xmode = log,
            log basis x = 2,
            xmin = 1, xmax = 8192,
            ymin = 175, ymax = 190,
            xticklabel = {\xinttheiexpr2^\tick\relax},
            log ticks with fixed point,
            ymajorgrids = true,
            xmajorgrids = true,
            grid style = dashed,
            legend cell align = left,
            legend pos = south east,
            no markers,
            every axis plot/.append style = {very thick},
            width = 0.75\textwidth,
            scale only axis,
            cycle list name = exotic,
            /pgf/number format/.cd, use comma
        ]
            \addplot table [x = blocks_per_sm, y = throughput-768,
                            col sep = semicolon]
                           {data/zcopy-nvidia-cuda-k20x-optimized-w.csv};
            \addlegendentry{CUDA} 

            \addplot table [x = blocks_per_sm, y = throughput-768,
                            col sep = semicolon]
                           {data/zcopy-nvidia-hip-k20x-optimized-w.csv};
            \addlegendentry{HIP} 

            \addplot table [x = blocks_per_sm, y = throughput-768,
                            col sep = semicolon]
                           {data/zcopy-nvidia-sycl-k20x-optimized-w.csv};
            \addlegendentry{SYCL} 
        \end{axis}
    \end{tikzpicture}
    \caption{K20x-Bandbreite für 768er-Blöcke (Schreiben)}
    \label{nvidia:zcopy:k20xexts}
\end{figure}

Die K20x-GPU erreicht etwa 75\% der theoretisch möglichen Bandbreite (siehe
Abbildung~\ref{nvidia:zcopy:k20xmax}). Dies ist durch das aktivierte ECC und
die auf dem Taurus-System nicht steuerbaren Taktraten der GPU zu erklären. Für
den Reduction-Benchmark wird daher eine tatsächlich erreichbare Obergrenze von
\SI{188}{\gibi\byte\per\second} angenommen.

\begin{figure}
    \centering
    \begin{tikzpicture}
        \begin{axis}[
            title = {zcopy -- optimiert -- K20x},
            xlabel = {Blöcke pro Multiprozessor},
            ylabel = {Bandbreite [\si{\gibi\byte\per\second}]},
            xmode = log,
            log basis x = 2,
            xmin = 1, xmax = 8192,
            ymin = 70, ymax = 260,
            xticklabel = {\xinttheiexpr2^\tick\relax},
            log ticks with fixed point,
            ymajorgrids = true,
            xmajorgrids = true,
            grid style = dashed,
            legend cell align = left,
            legend pos = south east,
            no markers,
            every axis plot/.append style = {very thick},
            width = 0.75\textwidth,
            scale only axis,
            cycle list name = exotic,
            extra y ticks = 249.6,
            extra y tick labels = {$249.6$},
            /pgf/number format/.cd, use comma
        ]
            \addplot table [x = blocks_per_sm, y = throughput-768,
                            col sep = semicolon]
                           {data/zcopy-nvidia-cuda-k20x-optimized-rw.csv};
            \addlegendentry{Lesen+Schreiben} 

            \addplot table [x = blocks_per_sm, y = throughput-768,
                            col sep = semicolon]
                           {data/zcopy-nvidia-cuda-k20x-optimized-w.csv};
            \addlegendentry{Schreiben} 

            \addplot table [x = blocks_per_sm, y = peak, col sep = semicolon]
                           {data/zcopy-nvidia-k20x-peak.csv};
            \addlegendentry{Peak}
        \end{axis}
    \end{tikzpicture}
    \caption{Theoretische und praktische K20x-Bandbreite für 768er-Blöcke}
    \label{nvidia:zcopy:k20xmax}
\end{figure}

\subsection{Reduction}

\subsection{N-Body}

Ein gesonderter Vergleich ist zwischen CUDA und SYCL nötig, da das
experimentelle Backend für NVIDIA-GPUs der ComputeCpp-Implementierung die
Funktion \texttt{rsqrtf} für die reziproke Quadratwurzel nicht unterstützt. Da
die äquivalente Berechnung \texttt{1.f / sqrtf} deutlich langsamer ist, wurde
stattdessen eine schnellere, weniger genaue Implementierung der Funktion
\texttt{rsqrtf} aus dem Quelltext des Spiels \textit{Quake 3 Arena} übernommen
(siehe Quelltext~\ref{nvidia:nbody:qrsqrt} für eine normale
C++-Implementierung).

\begin{code}
    \begin{minted}[fontsize=\small]{c++}
auto Q_rsqrt(float number) -> float
{
    auto x2 = number * 0.5f;
    auto y = number;
    auto i = *(reinterpret_cast<std::int32_t*>(&y));

    i = 0x5f3759df - (i >> 1);

    y = *(reinterpret_cast<float*>(&i));
    y *= 1.5f - (x2 * y * y);

    return y;
}
    \end{minted}
    \caption{Quake-3-Implementierung der rsqrt-Funktion}
    \label{nvidia:nbody:qrsqrt}
\end{code}
