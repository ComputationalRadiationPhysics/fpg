\section{Messungen auf NVIDIA-GPUs}
\label{nvidia}

\subsection{Verwendete Hard- und Software}

Die hier gezeigten Benchmark-Ergebnisse wurden auf verschiedenen Knoten des
\gls{hpc}-Systems Taurus gemessen:

\begin{itemize}
    \item Ein Knoten der gpu2 Partition mit vier GPUs des Modells Tesla K80
    \item Einem experimentellen Knoten mit einer GPU des Modells Tesla V100 
\end{itemize}

Die Messungen fanden innerhalb der SCS5-Umgebung statt. Das verwendete
CUDA-Modul war XXX. HIP wurde in folgender Form verwendet: XXX. Als
SYCL-Implementierung kam ComputeCpp der Firma Codeplay in der Version XXX zum
Einsatz.

\subsection{zcopy}

\subsection{Reduction}

\subsection{N-Body}

Ein gesonderter Vergleich ist zwischen CUDA und SYCL nötig, da das
experimentelle Backend für NVIDIA-GPUs der ComputeCpp-Implementierung die
Funktion \texttt{rsqrtf} für die reziproke Quadratwurzel nicht unterstützt. Da
die äquivalente Berechnung \texttt{1.f / sqrtf} deutlich langsamer ist, wurde
stattdessen eine schnellere, weniger genaue Implementierung der Funktion
\texttt{rsqrtf} aus dem Quelltext des Spiels \texttt{Quake 3 Arena} übernommen.
